\section*{Brief summary on how to write a report}

Here follows a brief recipe and recommendation on how to write a report for each
project.
\begin{itemize}
\item Give a short description of the nature of the problem and the eventual 
numerical methods you have used.
\item Describe the algorithm you have used and/or developed. Here you may find it convenient
to use pseudocoding. In many cases you can describe the algorithm
in the program itself.

\item Include the source code of your program. Comment your program properly.
\item If possible, try to find analytic solutions, or known limits
in order to test your program when developing the code.
\item Include your results either in figure form or in a table. Remember to
       label your results. All tables and figures should have relevant captions
       and labels on the axes.
\item Try to evaluate the reliabilty and numerical stability/precision
of your results. If possible, include a qualitative and/or quantitative
discussion of the numerical stability, eventual loss of precision etc. 

\item Try to give an interpretation of you results in your answers to 
the problems.
\item Critique: if possible include your comments and reflections about the 
exercise, whether you felt you learnt something, ideas for improvements and 
other thoughts you've made when solving the exercise.
We wish to keep this course at the interactive level and your comments can help
us improve it. We do appreciate your comments. 
\item Try to establish a practice where you log your work at the 
computerlab. You may find such a logbook very handy at later stages
in your work, especially when you don't properly remember 
what a previous test version 
of your program did. Here you could also record 
the time spent on solving the exercise, various algorithms you may have tested
or other topics which you feel worthy of mentioning.
\end{itemize}

\section*{Format for electronic delivery of report and programs}
The preferred format for the report is a PDF file and a link to your online Git repository, see the course web-page for more information. As programming language we prefer that you use C++. 

Finally, we recommend that you work together. Optimal working groups consist of 
2-3 students, but more people can collaborate. You can then hand in a common report. 

\section*{Literature}
\begin{enumerate}
\item B.~L.~Hammond, W.~A.~Lester and P.~J.~Reynolds, Monte Carlo methods
in Ab Inition Quantum Chemistry, World Scientific, Singapore, 1994, chapters
2-5 and appendix B.

\item B.H.~Bransden and C.J.~Joachain, Physics of Atoms and molecules,
Longman, 1986. Chapters 6, 7 and 9.
\item S.A.~Alexander and R.L.~Coldwell,
Int.~Journal of Quantum Chemistry, {\bf 63} (1997) 1001.  This article is available 
at the webpage of the course as the file jastrow.pdf under the project 1 link.
\item C.J.~Umrigar, K.G.~Wilson and J.W.~Wilkins, Phys.~Rev.~Lett.~{\bf 60}
(1988) 1719. 

\end{enumerate}